\documentclass[ignorenonframetext]{beamer}
\usepackage{hyperref}

\definecolor{rrblitbackground}{rgb}{0.55, 0.3, 0.1}

\newenvironment{rtbliteral}{

\begin{ttfamily}

\color{rrblitbackground}

}{

\end{ttfamily}

}

\usetheme{Warsaw}

\setbeameroption{hide notes}

% generated by Docutils <http://docutils.sourceforge.net/>
\usepackage{fixltx2e} % LaTeX patches, \textsubscript
\usepackage{cmap} % fix search and cut-and-paste in Acrobat
\usepackage{ifthen}
\usepackage[T1]{fontenc}
\usepackage[utf8]{inputenc}
\usepackage{textcomp} % text symbol macros
\newcommand{\DUfooter}{

}
\newcommand{\DUheader}{

}

%%% Custom LaTeX preamble
\usepackage{amssymb,amsmath}
\usepackage{ifxetex,ifluatex}
\usepackage{fixltx2e} % provides \textsubscript
\usepackage{ulem}
\usepackage[utf8]{inputenc}
\usepackage{enumerate}
\usepackage{graphicx}
% Comment these out if you don't want a slide with just the
% part/section/subsection/subsubsection title:
\AtBeginPart{\frame{\partpage}}
\AtBeginSection{\frame{\sectionpage}}
\AtBeginSubsection{\frame{\subsectionpage}}
\AtBeginSubsubsection{\frame{\subsubsectionpage}}
\setlength{\parindent}{0pt}
\setlength{\parskip}{6pt plus 2pt minus 1pt}
\setlength{\emergencystretch}{3em}  % prevent overfull lines
\setcounter{secnumdepth}{0}




\title{Aw2D-Android}
\author{Ferdinand Majerech}
\institute[UPJŠ]{Univerzita Pavla Jozefa Šafárika v Košiciach\\UPJŠ}

%%% User specified packages and stylesheets

%%% Fallback definitions for Docutils-specific commands

% hyperlinks:
\ifthenelse{\isundefined{\hypersetup}}{
  \usepackage[colorlinks=true,linkcolor=blue,urlcolor=blue]{hyperref}
  \urlstyle{same} % normal text font (alternatives: tt, rm, sf)
}{}


%%% Body
\begin{document}
NDK, Necessitas and Aw2D-Android© Ferdinand Majerech, 2012-2013
\begin{frame}[plain]
  \titlepage
\end{frame}

\begin{frame}[fragile]
\frametitle{NDK}

\begin{itemize}

\item Native development kit for Android

\item Allows development in compiled languages
- C, C++ by default

\item Reason: speed

\item Designed to \emph{extend} apps written for Dalvik

\item .so accessed through JNI from Java
\end{itemize}
\end{frame}

\begin{frame}[fragile]
\frametitle{C/C++ in NDK}

\begin{itemize}

\item By default, no C++ exceptions, no RTTI, no STL
\begin{itemize}

\item Can be enabled (with multiple STL implementations).
\end{itemize}

\item Dynamic linking sucks
\begin{itemize}

\item Static ctors called twice, static dtors never called
\end{itemize}

\item Very few libs guaranteed to be present
\begin{itemize}

\item Zlib, dl, EGL+GLES1/2, SLES, jnigraphics, OpenMAX

\item Native activity stuff (with a C API as readable as libPNG)
\begin{itemize}

\item Allows ``fully native'' code, i.e. JNI is needed for anything useful
\end{itemize}
\end{itemize}

\item Need JNI to access most APIs.
\begin{itemize}

\item Some frameworks exist, better and worse
\end{itemize}
\end{itemize}
\end{frame}

\begin{frame}[fragile]
\frametitle{OpenGL ES 2.0 in practice}

\begin{itemize}

\item Cleaned up spec - no immediate mode, etc.

\item Drivers suck (everywhere), not in line with spec

\item NPOT textures silently fail (Tegra)

\item onPause wastes GL context, or \emph{maybe} wastes it
\end{itemize}
\end{frame}

\begin{frame}[fragile]
\frametitle{GLSL ES}

\begin{itemize}

\item No const arrays

\item Arrays in spec, might or might not work on a device

\item Float precision specifiers, affect performance:
\begin{itemize}

\item highp: 16bit spec, 31bit common - vertices

\item mediump: 10bit spec, 15bit common - bounded distances

\item lowp: 8bit spec and used in practice - colors, normals
\end{itemize}
\end{itemize}
\setbeamerfont{quote}{parent={}}
%
\begin{quote}{\ttfamily \raggedright \noindent
lowp~vec3~pointLighting(in~int~light,\\
~~~~~~~~~~~~~~~~~~~~~~~~in~lowp~vec3~pixelDiffuse,\\
~~~~~~~~~~~~~~~~~~~~~~~~in~lowp~vec3~pixelNormal,\\
~~~~~~~~~~~~~~~~~~~~~~~~in~highp~vec3~pixelPosition);
}
\end{quote}
\setbeamerfont{quote}{parent=quotation}
\end{frame}

\begin{frame}[fragile]
\frametitle{ABI}

\begin{itemize}

\item armeabi - ARMv5+, no native floating point
\begin{itemize}

\item useless for games
\end{itemize}

\item armeabi-v7a - ARMv7, has hardfp

\item SIMD not guaranteed to be present
\begin{itemize}

\item Must check for NEON at runtime

\item Need separate binaries for devices with and without NEON
\end{itemize}
\end{itemize}
\end{frame}

\begin{frame}[fragile]
\frametitle{TLDR}

\begin{itemize}

\item Need separate binaries for ABIs and ABI extensions

\item Native is a royal PITA compared to Linux/Windows

\item Do not want to use NDK directly
\end{itemize}
\end{frame}

\begin{frame}[fragile]
\frametitle{Necessitas}

\begin{itemize}

\item Qt for Android

\item Beta - no stable yet(To be stable in Qt 5.2)

\item Contains Qt creator modified for Android

\item Similar to ADT - fast build/debug cycle

\item Generates all Java wrapper code

\item Development in C++ or QML/Javascript (or both)

\item Qt project file takes role of manifest
\end{itemize}
\end{frame}

\begin{frame}[fragile]
\frametitle{Qt}

\begin{itemize}

\item C++ framework similar to .NET/Java libraries

\item Low overhead mid-level OpenGL wrappers

\item QML for declarative apps with Javascript

\item Supports \emph{some} phone features
\begin{itemize}

\item Qt was built for Maemo/Meego/Meltemi/Mer/jolla
\end{itemize}

\item Cross-platform (Android, BB10, Jolla, Win/Mac/Linux ...)
\begin{itemize}

\item iOS in works (well, Android, too)
\end{itemize}
\end{itemize}
\end{frame}

\begin{frame}[fragile]
\frametitle{Necessitas is not stable yet}

\begin{itemize}

\item Some stuff must be hand-edited in manifest
\begin{itemize}

\item Will get overwritten when generated... URRGH
\end{itemize}

\item Android lifecycle is handled on background
\begin{itemize}

\item Qt event loop stops

\item If timing outside the Qt event loop, no onPause/onStop for you

\item Qt ensures GL context does not get murdered

\item No way to onDestroy yet (Qt has an API for it - not implemented yet)
\end{itemize}

\item If a feature is not supported, need to use JNI
\begin{itemize}

\item Clashing with generated code
\end{itemize}
\end{itemize}
\end{frame}

\begin{frame}[fragile]
\frametitle{Necessitas apps}

\begin{itemize}

\item Ministro

\item At first launch, asks user to download the needed subset of Qt

\item Gets correct builds for Android version and ABI on device

\item Libs shared between Necessitas apps

\item Other libs - must pack with app, built for correct ABI
\end{itemize}
\end{frame}

\begin{frame}[fragile]
\frametitle{Aw2D-Android}

\begin{itemize}

\item Awesome2D for Android

\item Proof-of-concept of 3D lighting for 2D graphics

\item In NDK/C++/Necessitas, written specifically for Android/ARM

\item Eventually to be cannibalized into an even more cross-platform Awesome2D in D
\end{itemize}
\end{frame}

\begin{frame}[fragile]
\frametitle{Awesome2D}

\begin{itemize}

\item Dynamic ``3D'' lighting in 2D graphics

\item Using colors/normals/positions pre-rendered from 3D data

\item Reconstructing 3D data per pixel

\item Using (modified) Blinn-Phong lighting

\item Graphics comparable to 3D isometric games (Diablo 3)

\item Low vertex, medium-high fragment overhead
\end{itemize}
\end{frame}

\begin{frame}[fragile]
\frametitle{Current state}

\begin{itemize}

\item Basic 3D lighting demo/benchmark.

\item The only interactivity is the ability to move two lights.

\item Brute force approach at the moment (no texture packing, etc.)

\item Surprisingly good performance (considering brute-force)
\begin{itemize}

\item But more benchmarking work is needed
\begin{itemize}

\item Esp. map loading
\end{itemize}
\end{itemize}
\end{itemize}
\end{frame}

\begin{frame}[fragile]
\frametitle{Implementation}

\begin{itemize}

\item Basic sprites
\begin{itemize}

\item No facings yet (need a YAML parser (yaml-cpp?))
\end{itemize}

\item GLSL ES shaders doing the lighting

\item Directional, point light sources

\item Uniform wrapper to avoid excessive reuploading
\begin{itemize}

\item More readable uniform management; to be backported
\end{itemize}

\item Simple 2D camera
\end{itemize}
\end{frame}

\begin{frame}[fragile]
\frametitle{TODO}

\begin{itemize}

\item Full sprite parsing

\item Map loading

\item Optimizations from desktop Awesome2D

\item More ARM-specific optimizations
\begin{itemize}

\item 16-bit texcoords, vertex \& buffer alignment,
\end{itemize}

\item Test more devices/use device specific GL extensions
\end{itemize}
\end{frame}

\begin{frame}[fragile]
\frametitle{Thank you for your attention!}

\end{frame}

\end{document}
